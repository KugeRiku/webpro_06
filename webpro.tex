\documentclass[uplatex,dvipdfmx]{jsarticle}

\usepackage[uplatex,deluxe]{otf} % UTF
\usepackage[noalphabet]{pxchfon} % must be after otf package
\usepackage{stix2} %欧文&数式フォント
\usepackage[fleqn,tbtags]{mathtools} % 数式関連 (w/ amsmath)
\usepackage{hira-stix} % ヒラギノフォント&STIX2 フォント代替定義(Warning回避)
\usepackage{listings}

\begin{document}

\title{webプログラミングレポート}
\author{24G1052 久家力孔}
\maketitle
\section{ソースコード}
https://github.com/KugeRiku/webpro\_06.git

\section{利用者向け仕様書}

\subsection{サービス概要}
本サービスでは,ユーザーが名前とメッセージを入力し,掲示板に投稿することができる.
また表1に記した,ユーザエクスペリエンスを向上させるための三つの追加機能がある.

\begin{table}[ht]
    \centering
    \caption{本サービスにおける機能}
    \begin{tabular}{|c|c|}
        \hline
        投稿の検索 & キーワードを基に投稿を検索し、関連する結果のみを表示する \\ \hline
        新規投稿の通知 &  新しい投稿があれば通知する \\ \hline
        「いいね」機能 & 自分が気に入った投稿に「いいね」ボタンを押すことができる \\ \hline
    
    \end{tabular}
\end{table}

\subsection{画面のレイアウト}
表2に,本サービスの基本的な画面構成を記した.

\begin{table}[ht]
    \centering
    \caption{画面のレイアウトについて}
    \begin{tabular}{|c|c|}
        \hline
        投稿フォーム & ページ上部に、名前と新規投稿の入力欄、送信ボタンがある \\ \hline
        投稿一覧 &  ページの左上部に表示されていく \\ \hline
        検索バー & ページ中央部分に、検索したいキーワードの入力欄がある \\ \hline
    
    \end{tabular}
\end{table}

\subsection{サービス利用の流れ}
メッセージを投稿するためには,名前と投稿したいメッセージを入力欄に入力する.
そして入力欄の右部にある送信ボタンを押すことで、メッセージが掲示板に投稿される.
投稿フォーム下部の投稿チェックボタンを押すことで,新しく投稿されたメッセージを確認することができる.
検索バーにキーワードを入力してから検索ボタンを押すことで,入力したキーワードのみを含むメッセージを表示することが
できる.

\section{管理者向け仕様書}

\subsection{サービス概要}
この掲示板サービスは,Node.jsベースで構築されたWebアプリケーションであり,四つの主要機能が存在する.
\begin{itemize}
    \item ユーザーが投稿を送信、閲覧できる機能.
    \item 投稿をキーワードで検索できる機能.
    \item 新規投稿を通知する機能.
    \item 各投稿に対して「いいね」を付けることができる機能.
\end{itemize}

\subsection{必要ソフトウェア}
サービスを立ち上げるために以下のソフトウェアが必要である
\begin{itemize}
    \item Node.js
    \item npm(Node.jsに同梱)
    \item Google Chrome, Firefox, Safariのいずれか
\end{itemize}

\subsection{セットアップ手順}
まず,プロジェクトのディレクトリに以下のファイルを配置する
\begin{itemize}
    \item app7.js(サーバースクリプト)
    \item bbs.html(HTML構造ファイル)
    \item bbs.js(クライアントサイドスクリプト)
    \item bbs.css(スタイルシート)
\end{itemize}

次に依存ライブラリをインストールするため、以下のコマンドを実行する
\begin{verbatim}
npm install express ejs
\end{verbatim}

そして,以下のコマンドを実行してサーバーを起動する
\begin{verbatim}
node app7.js
\end{verbatim}

成功時,次のメッセージが表示される
\begin{verbatim}
Example app listening on port 8080!
\end{verbatim}

\section{開発者向け仕様書}

\subsection{サービス概要}
この掲示板サービスは,Node.jsベースで構築されたWebアプリケーションである.
サーバーサイドではexpressフレームワークを利用してHTTPリクエストを処理し,
クライアントサイドでは,JavaScriptによる投稿の送信,検索,通知機能を実現した.

\subsection{サーバーサイド(\texttt{app7.js})の構造と変数の説明}
変数について
\begin{itemize}
    \item \texttt{bbs}:配列形式で掲示板の投稿データを保持。投稿は例として以下のような構造を持つ
    \begin{verbatim}
    { name: "Riku", message: "Hello", likes: 3 }
    \end{verbatim}
    \item \texttt{app}:Expressのインスタンスを保持
\end{itemize}

投稿の送信 (\texttt{/post})
\begin{itemize}
    \item メソッド:POST
    \item データ形式の名前:\texttt{application/x-www-form-urlencoded}
    \item 送信データの具体例
    \begin{verbatim}
    name=Riku&message=Hello
    \end{verbatim}
    \item 説明:クライアントから送信された\texttt{name}と\texttt{message}をサーバーの掲示板データに追加
    \item レスポンスデータの形式
    \begin{itemize}
        \item データ形式の名前:JSON
        \item レスポンスデータの具体例
        \begin{verbatim}
        { "number": 3 }
        \end{verbatim}
        \item 説明:サーバーに保存されている投稿の総数(\texttt{number})を返す
    \end{itemize}
\end{itemize}

投稿の取得 (\texttt{/read})
\begin{itemize}
    \item メソッド:POST
    \item データ形式の名前:\texttt{application/x-www-form-urlencoded}
    \item 送信データの具体例
    \begin{verbatim}
    start=10
    \end{verbatim}
    \item 説明:\texttt{start}で指定したインデックス以降の投稿をサーバーから取得
    \item レスポンスデータの形式
    \begin{itemize}
        \item データ形式の名前:JSON
        \item レスポンスデータの具体例
        \begin{verbatim}
        {
            "messages": [
                { "name": "Riku", "message": "Hello", "likes": 3 },
                { "name": "Sora", "message": "Hi there!", "likes": 0 }
            ]
        }
        \end{verbatim}
        \item 説明:取得された投稿データを配列として返す
    \end{itemize}
\end{itemize}

新規投稿数の確認 (\texttt{/check})
\begin{itemize}
    \item メソッド:POST
    \item データ形式の名前:\texttt{application/x-www-form-urlencoded}
    \item 送信データ:なし
    \item 説明:現在の投稿数を取得し、クライアントに通知
    \item レスポンスデータの形式
    \begin{itemize}
        \item データ形式の名前:JSON
        \item レスポンスデータの具体例
        \begin{verbatim}
        { "number": 15 }
        \end{verbatim}
        \item 説明:掲示板に存在する投稿の総数を返す
    \end{itemize}
\end{itemize}

検索機能 (\texttt{/search})
\begin{itemize}
    \item メソッド:POST
    \item データ形式の名前:\texttt{application/x-www-form-urlencoded}
    \item 送信データの具体例
    \begin{verbatim}
    keyword=Hello
    \end{verbatim}
    \item 説明:指定された\texttt{keyword}を含む投稿を検索し、結果を返す
    \item レスポンスデータの形式
    \begin{itemize}
        \item データ形式の名前:JSON
        \item レスポンスデータの具体例
        \begin{verbatim}
        {
            "results": [
                { "name": "Riku", "message": "Hello", "likes": 3 }
            ]
        }
        \end{verbatim}
        \item 説明:\texttt{keyword}を含む投稿データを配列として返す
    \end{itemize}
\end{itemize}

いいね機能 (\texttt{/like})
\begin{itemize}
    \item メソッド:POST
    \item データ形式の名前:\texttt{application/x-www-form-urlencoded}
    \item 送信データの具体例
    \begin{verbatim}
    id=0
    \end{verbatim}
    \item 説明:指定された投稿(\texttt{id}で識別)の「いいね」数をインクリメント
    \item レスポンスデータの形式
    \begin{itemize}
        \item データ形式の名前:JSON
        \item レスポンスデータの具体例
        \begin{verbatim}
        { "success": true, "likes": 4 }
        \end{verbatim}
        \item 説明:
        \begin{itemize}
            \item \texttt{success}:処理が成功したかを示す(\texttt{true}または\texttt{false})。
            \item \texttt{likes}:投稿の更新後の「いいね」数。
        \end{itemize}
    \end{itemize}
\end{itemize}

\subsection{クライアントサイド(\texttt{bbs.js})の構造と変数の説明}
変数について
\begin{itemize}
    \item \texttt{number}:現在表示中の投稿数を管理
    \item \texttt{lastCheckedNumber}:最後にチェックした投稿数を記録
\end{itemize}

主な関数とその役割
\begin{itemize}
    \item \texttt{renderPost(post, id)}:投稿データをHTML要素として生成し、掲示板に追加
    \item \texttt{loadPosts(start)}:指定インデックス以降の投稿をサーバーから取得し、表示
    \item \texttt{setInterval}:一定間隔で新規投稿数をチェックし、通知
\end{itemize}

\subsection{データのフロー}
以下にクライアントからサーバーへのリクエストの処理フローを示す

\begin{enumerate}
    \item クライアントが\texttt{/post}エンドポイントにデータを送信(例:\texttt{name=Riku\&message=Hello})
    \item サーバーは\texttt{express.urlencoded}でデータを解析し、\texttt{req.body}に保存
    \item \texttt{bbs.push()}を使い、投稿データを\texttt{bbs}配列に追加
    \item 処理が成功すると、サーバーは投稿数(例:\texttt{\{"number": 3\}})をJSON形式で返す
\end{enumerate}

同様に,\texttt{/read},\texttt{/search},\texttt{/like}などのエンドポイントも,リクエストデータを処理し,必要なレスポンスを生成する.

\subsection{独自使用技術の概要と採用理由}
\texttt{DOMContentLoaded}
\begin{itemize}
    \item \textbf{概要}:HTMLの解析が完了し、DOMツリーの読み込み完了後に発火するイベント
    \item \textbf{採用理由}:HTML構造が準備されていない状態でスクリプトが実行されるエラーを防ぐため
\end{itemize}

\texttt{forEach}
\begin{itemize}
    \item \textbf{概要}:配列の各要素に対して、一つずつ処理を行うためのメソッド
    \item \textbf{採用理由}:シンプルで可読性が高く、ループ処理を簡潔に記述できるため
    \item \textbf{例}
    \begin{verbatim}
    response.messages.forEach((post, id) => renderPost(post, id));
    \end{verbatim}
\end{itemize}

テンプレートリテラル(\texttt{\$\{response.likes\}})
\begin{itemize}
    \item \textbf{概要}:バッククォート(\texttt{`})で囲んだ文字列中に変数や式を埋め込む構文
    \item \textbf{採用理由}:ボタンの表示内容を分かりやすく設定するため
    \item \textbf{例(「いいね」の記号はpdfに表示されないので'b'で代用)}
    \begin{verbatim}
    likeButton.innerText = `b ${response.likes}`;
    \end{verbatim}
\end{itemize}

\texttt{setInterval}
\begin{itemize}
    \item \textbf{概要}:一定時間ごとに指定した関数を繰り返し実行する
    \item \textbf{採用理由}:リアルタイムでの新規投稿チェックを実現するため
    \item \textbf{例}:
    \begin{verbatim}
    setInterval(() => {
        fetch("/check", { method: "POST" })
            .then((response) => response.json())
            .then((data) => alert(`新しい投稿が ${data.number} 件あります`));
    }, 5000);
    \end{verbatim}
\end{itemize}

\end{document}